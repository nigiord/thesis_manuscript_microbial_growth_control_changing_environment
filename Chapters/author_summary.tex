\section*{Author Summary}
Microbial growth is the process by which cells sustain and reproduce themselves from available matter and energy.
Strategies enabling microorganisms to optimize their growth rate have been extensively studied, but mostly in stable environments.
Here, we build a coarse-grained model of microbial growth and use methods from optimal control theory to determine a resource allocation scheme that would lead to maximal biomass accumulation when the cells are dynamically shifted from one growth medium to another.
We compare this optimal solution with several cellular implementations of growth control, based on the capacity of the cell to sense different physiological variables.
We find that strategies maximizing growth in steady-state conditions perform quite differently in dynamical conditions.  
Moreover, the control strategy with performance close to the theoretical maximum exploits information of more than one physiological variable, suggesting that optimization of microbial growth in dynamical rather than steady environments requires broader sensory capacities.
Interestingly, the ppGpp alarmone system in the enterobacterium \textit{Escherichia coli}, known to play an important role in growth control, has structural similarities with the control strategy approaching the theoretical maximum.
It senses a discrepancy between the concentrations of precursors and ribosomes, and adjusts ribosome synthesis in an on-off fashion.
This suggests that \textit{E. coli} is adapted for environments with intermittent, rapid changes in nutrient availability.