\chapter{Monitoring Gene Expression Machinery abundance during a glucose upshift}

\section{Introduction}

\begin{itemize}
\item Understanding growth transitions is important, because so and so.
\item We predicted earlier counter-intuitive bang-bang-singular behaviours for GEM during transition
\item But we're missing experimental data on what really happens in real cells during these transitions (quick review of available data)
\item We fix it by making a strain with fluorescent ribosomes and monitoring ribosome abundance during growth transitions in a microfluidic device.
\end{itemize}

\section{Material and Methods}

\subsection*{Bacterial strain construction}

To achieve dynamical quantification of ribosome abundance, we designed and constructed a strain containing a translational fusion of \textit{gfpmut2} [citation Alon] to the C-terminus of rpsB (the gene coding the ribosome subunit S2).
This design was inspired by the work of Bakshi \textit{et al}, who used a similar construction to assess the quantitative spatial distribution of ribosomes in living \textit{E. coli}~\cite{bakshi_superresolution_2012}.

The transcription factor tsf is under the control of the same promoter than rpsB, and is located after the C-terminus end.
In order to ensure that tsf expression was not affected by our translation fusion of rpsB, we used a double-selection procedure to get rid of every resistance gene that the construction might introduce.
This is something that was not done in the original construction from Bakshi \textit{et al}~\cite{bakshi_superresolution_2012}

A DNA fragment containing a double selection cassette was amplified with two long primers annealing respectively to the region just after the STOP codon at the C-terminus rpsB, and to the 3'-end of gfpmut2.
The double-selection cassette contained a resistance gene to Kanamycin (positive selection) and a gene coding for the CcdB toxin under the control of the P\textsubscript{BAD} promoter (negative selection in presence of Arabinose).
This cassette is referred below as kan-P\textsubscript{BAD}-ccdB.

Another DNA fragment containing an ATG-less gfpmut2~\cite{zaslaver_comprehensive_2006} was amplified with long primers annealing respectively to the C-terminus region of rpsB (just before the stop codon), and the 5'-end of the kan-P\textsubscript{BAD}-ccdB cassette.
The first primer also contained a 18-bp (base pair) linker that was selected according to the method section of~\cite{bakshi_superresolution_2012}.

Both fragments (the gfpmut2 reporter and the kan-P\textsubscript{BAD}-ccdB cassette) were assembled using Gibson assembly~[citation].
Both PCR products were quantified using NanoDrop and mixed in equimolar proportions with a commercial Gibson Assembly Master mix.
A final product of 2683 bp was obtained:
\begin{itemize}
\item *(50 bp) the C-terminus of rpsB without the STOP codon 
\item (18 bp) a linker
\item (714 bp) the gfpmut2 sequence without the initial ATG
\item (1851 bp) the kan-P\textsubscript{BAD}-ccdB
\item *(50 bp) the region directly after rpsB in \textit{E. coli}
\end{itemize}
Regions labeled with * are supposed to anneal with the \textit{E. coli} chromosome.
The complete sequence of this fragment, as well as all the primers used are available in the Supporting Information of this chapter.

This fragment was electroporated in a BW25113 background strain containing the pSIM5 plasmid providing Chloramphenicol resistance (lambda-red recombinaison).
A kanamycin-resistant colony was selected and verified to exhibit green fluorescence in a Tecan microplate reader.
A new 100-bp fragment containing 50 bp of the 3'-end of gfpmut2 and 50 bp of the region just after rpsB was electroporated into this strain (sequence available in Supporting Information) to get rid of the kan-P\textsubscript{BAD}-ccdB cassette.
An arabinose-resistant colony was selected and verified to be kanamycin-sensitive and to still exhibit green fluorescence.
Finally, the strain was grown overnight at 42$^\circ$C to get rid of the pSIM5 plasmid and a chloramphenicol-sensitive colony was selected.
The region after rpsB was sequenced and determined to be as anticipated (full sequence available in Supporting Information).

In parallel, the same protocol was used to construct mCherry and cfp variants of the same strain.
However, we were only able to recombine and obtain the gfp and mCherry versions for reasons that were not investigated.
The full sequence of the final rpsB-mCherry strain is available in Supporting Information.

The rpsB-gfp and rpsB-mCherry strains were caracterised on different media using a Tecan microplate reader.
They were showed to exhibit a wild-type growth rate and sufficient fluorescence level to allow quantification.
However, the rpsB-mCherry strain was shown to exhibit strange fluorescence dynamics, especially during growth transitions (entering stationary phase), which made it unsuitable for our study.
Our effort were then concentrated on the rpsB-gfp strain.
Data as well as information on the matter are available in Supporting Information.

\subsection{Microfluidic device}

Mother machine because needed to change the medium, single cell data because asynchronization etc.

\subsection{Cell segmentation}

We used this and that to analyze the microscopic images.
We applied the following corrections.

\subsection{Kalman smoothing}

We used this and that to analyze the microscopic images.
We applied the following corrections.

\section{Results}

\subsection{Model calibration}

We want to get $\alpha (t)$ during a transition. We show that $\alpha$ can be obtained if we measure so and so.

\subsection{Bang-bang expression during transitions, as predicted}

One stunning figure that show the RFU / cell / pixels, and the predicted $\alpha$ for the glucose-acetate transition.

\subsection{Discrepancies with the model and model adaptation}

The growth rate stalls after the addition of glucose, which was not originally predicted by the model.
This is because so and so.
This can be taken into account if we add another sector (M1 and M2).
This raises a new interesting question : during the transition amino-acids drop, so the cell should produce ribosomes, but it also need to produce the new enzymes to actually be able to perform metabolism.
We thus have competition between several layers of regulation, which can be added to the model by doing so and so.

\section{Discussion}

More research need to be done.

Could be interesting to use transitions for syn bio purposes.
Could be interesting to validate such results in different microorganisms, with different type of medium (fast env, slow env, etc)

\section{Supporting Information}

\begin{itemize}
\item Strain validation (signal-to-noise ratio, growth rate not hampered in microplate reader)
\item Primers used for gibson assembly
\item Difference between mCherry and GFP (mCherry not good because so and so, GFP was chosen)
\end{itemize}