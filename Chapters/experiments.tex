\chapter{Monitoring Gene Expression Machinery abundance during a glucose upshift}

\section{Introduction}

\begin{itemize}
\item Understanding growth transitions is important, because so and so.
\item We predicted earlier counter-intuitive bang-bang-singular behaviours for GEM during transition
\item But we're missing experimental data on what really happens in real cells during these transitions (quick review of available data)
\item We fix it by making a strain with fluorescent ribosomes and monitoring ribosome abundance during growth transitions in a microfluidic device.
\end{itemize}

\section{Methods}

\subsection{Strain design}

We used rpsB, GFP, and a chromosomic construction because so and so.

\subsection{Strain construction}

We used Gibson assembly, Kan-pbad-ccdb, etc and obtain the following sequence.

\subsection{Microfluidic device}

Mother machine because needed to change the medium, single cell data because asynchronization etc.

\subsection{Cell segmentation}

We used this and that to analyze the microscopic images.
We applied the following corrections.

\subsection{Kalman smoothing}

We used this and that to analyze the microscopic images.
We applied the following corrections.

\section{Results}

\subsection{Model calibration}

We want to get $\alpha (t)$ during a transition. We show that $\alpha$ can be obtained if we measure so and so.

\subsection{Bang-bang expression during transitions, as predicted}

One stunning figure that show the RFU / cell / pixels, and the predicted $\alpha$ for the glucose-acetate transition.

\subsection{Discrepancies with the model and model adaptation}

The growth rate stalls after the addition of glucose, which was not originally predicted by the model.
This is because so and so.
This can be taken into account if we add another sector (M1 and M2).
This raises a new interesting question : during the transition amino-acids drop, so the cell should produce ribosomes, but it also need to produce the new enzymes to actually be able to perform metabolism.
We thus have competition between several layers of regulation, which can be added to the model by doing so and so.

\section{Discussion}

More research need to be done.

Could be interesting to use transitions for syn bio purposes.
Could be interesting to validate such results in different microorganisms, with different type of medium (fast env, slow env, etc)

\section{Supporting Information}

\begin{itemize}
\item Strain validation (signal-to-noise ratio, growth rate not hampered in microplate reader)
\item Primers used for gibson assembly
\item Difference between mCherry and GFP (mCherry not good because so and so, GFP was chosen)
\end{itemize}