\chapter{Introduction}

\textit{There's an infinity of things that have never been done before, and
most of those things are not worth doing.} -- Jeremy Fox, \textit{Dynamic Ecology}, 2016

\section*{Summary}

In this section...

\selectlanguage{french}
\section*{Résumé}

Dans cette section...
\selectlanguage{english}

\section{Context and motivations}

Most part of life beauty and diversity is hidden from us in the form of microorganisms.
These small factories are optimized to grab and use energy from their environment, in the form of organic matter, highly reactive chemicals, or even photons [citation trophic].
In that sense, microorganisms exhibit the most fundamental features of life.
They live almost everywhere, sometimes in the strangest possible places [citation space].
This nature of ubiquitous, extremely optimized living systems makes them perfect to unearth the fundamental forces that shape life as we know it.



They also managed to become the thoughest of all the living systems.
\textit{Their diversity allow them to colonize every biotope, but remarkably, a single species can live in broad conditions, growing on different carbon sources or resist to really harmful environments (even space!).
}

Their capacity to survive and exploit strong environmental fluctuations lay on another ubiquitous process of life as we know it: self-catalysis and gene expression.
\textit{Short explanation on DNA, RNA, Protein and the regulations behind, that can be seen as loading "programs" depending on the environment the organism as to cope with.}

Such a mechanism can be seen as a resource allocation problem.
\textit{Because they can build everything from a single carbon source.
But they have a limited protein production capacity, so choice have to be made for what need to be produced, natural selection selecting for the best strategies.}

General growth laws have been identified.
\textit{The study of resource allocation led to fundamental laws like ribosome allocation or overflow metabolism, processes that can only be understood through growth rate maximization.}

Biomass production is an important fitness factor of microorganisms.
\textit{Broadly used in a variety of study}

One of the key feature that made these works possible was to study the living system in a well-defined and reproducible state: balanced-growth.
\textit{We describe balanced-growth, what it means for the dynamical system, and how you can reproduce it in a lab.
We also explain why it is so convenient.}

The mechanisms identified exist to cope with environmental fluctuations in the natural habitat of microorganisms, but are being studied in constant environments.
\textit{Now we start to see the problem: there's a caveat in studying a system that is designed to respond to changes, if we focus on unchanging conditions.
And we are currently unable to create such beautiful laws when taking into account dynamical conditions.}

While adopting a dynamical perspective on resource allocation is challenging, it could be fundamental to the life-treat-history of microorganisms.
\textit{It is also a problem if the system we are studying poorly encountered this state during its evolution.
In other words, "maybe bacteria are confused during steady-state".
Hence, there is really a strong need for a mathematical framework to study resource allocation in dynamical conditions, and to know what actually happens during these growth transitions.}

\section{Problem statement}

How do microorganisms adjust their resource allocation during environmental transitions ?
\textit{As we discussed, I tried to be as specific as possible here.}

\section{Related work}

(the work on dynamic, Van der Berg, Alon, Levy, etc... the work on steady-state has normally been described previously. Or, I don't know, maybe I can shortly describe it in the context and extensively describe it here.)

\section{Approach}

We first tackle the problem of creating a mathematical framework for studying environmental transitions.
\textit{Building upon the work cited above, we create a simple self-replicator model that allocate resources between two sections.}

The goal is to identify what would be an ideal transition.
\textit{We all have an intuitive idea of what should be a good transition: it should be fast, robust, etc... But it is unclear how this relates to the fitness factor that shaped the regulatory mechanisms of microorganisms.}

We describe dynamical optimilaty by using optimal control theory.
\textit{Brieve explanation about optimal control theory and how it is used in economic or engineering to optimize or understant dynamical processes.}

Then, after we know what to look for, we try to actually measure how resource allocation occurs in a model organism: \textit{Escherichia coli}.
\textit{We justify the use of coli and introduce the gene expression fluorescent reporter.
We made a strain with fluorescent ribosomes, and studied our experimental framework in different transitions, at the population and the single cell level.
We justify the use of both by the fact that population is easy for "screening", and single cell allow for a better resolution and take care of the "desynchronization" biais that could occur.}

\section{Organisation of the dissertation}

(announcing the plan of the manuscript, which actually follow exactly the approach part, without the lenghty justifications)