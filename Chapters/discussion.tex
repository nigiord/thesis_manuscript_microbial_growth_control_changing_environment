\chapter{Discussion}
\label{chap:discussion}

\textit{"Apples fall onto the Earth because natural selection eliminated apples falling towards the sky."} -- @tomroud~\cite{tomroud_tom_2016}, original source unknown.

\begin{center}
\noindent\rule{4cm}{0.1pt}
\end{center}

\selectlanguage{french}
\begin{chapter_summary}{Discussion}
Contrairement à sa cousine la physique, la biologie manque encore de théories quantitatives à fort pouvoir prédictif.
Même si de grosses avancées ont eu lieu en biologie évolutive, d'autres domaines sont loin d'être aussi développés d'un point de vue mathématique.
C'est notamment le cas de la croissance, qui malgré la place fondamentale qu'elle occupe dans ce qui définit un être-vivant, repose sur des mécanismes qui semblent varier énormément d'un organisme à l'autre, et pour lesquels des lois fondamentales sont dures à identifier.

Chez les microorganismes, nous disposons néanmoins de lois de croissance.
Elles montrent de manière empirique que malgré la grande diversité des mécanismes moléculaires qui assurent le contrôle de la croissance, la composition des microorganismes obéit à des règles universelles lorsqu'ils se multiplient à l'état stationnaire dans différents environnements.
En particulier, l'abondance de leurs ribosomes s'ajuste linéairement avec la richesse du milieu, d'une manière qui maximise leur taux de croissance.
Mais comme décrit au cours du Chapitre~\ref{chap:introduction}, ces lois ont été essentiellement établies en croissance stationnaire, un état très rarement rencontré par ces organismes dans leur milieu naturel.
Pour quelles raisons les microorganismes seraient-ils optimisés pour un état qu'ils n'ont que très peu rencontré au cours de leur évolution ?
Les lois de croissance à l'état stationnaire ne seraient-elles pas une application particulière de lois plus générales qui s'appliquent en environnements variables ?

Notre but dans ce manuscrit a été d'établir un cadre de travail à la fois théorique et expérimental dans lequel une perspective dynamique peut être adoptée sur les lois de croissance.
Au cours du Chapitre~\ref{chap:theory}, nous avons utilisé un modèle d'auto-réplicateur pour évaluer la redistribution optimale des ressources lors d'un changement environnemental.
En élargissant les principes d'optimalité établis dans les lois de croissance stationnaire, nous avons montré qu'une distribution des ressources de type tout-ou-rien (\textit{bang-bang} en anglais) optimise la biomasse générée tant que la composition cellulaire n'est pas à l'équilibre.
Cela nous a servi de point de référence pour comparer entre elles différentes stratégies de régulation.
Ainsi, nous avons montré que les stratégies mesurant l'état interne de la cellule s'avèrent plus efficaces que celles tirant leur information de l'environnement, contrairement à l'état stationnaire de croissance où toutes ces stratégies sont strictement équivalentes.
De plus, une stratégie légèrement plus complexe mais proche du système ppGpp chez \textit{Escherichia coli}, est théoriquement capable de s'approcher de la distribution tout-ou-rien identifiée comme optimale.

La simplicité du modèle, même si elle peut rendre certains lecteurs sceptiques, s'est avérée cruciale dans notre étude.
En effet, cela nous a permis de clairement spécifier les hypothèses mises en jeu.
La plus critique s'avère être celle qui considère la production de biomasse comme étant le facteur de \textit{fitness} principal que l'on soit en environnement stable ou dynamique.
Bien que largement répandue, cette hypothèse n'est finalement basée que sur des arguments indirects, ou bien établis dans des conditions de laboratoire, et sur des souches de laboratoire (Chapitre~\ref{chap:introduction}).
De fait, en conditions naturelles et à l'échelle des temps évolutifs, il n'est pas certain que ce critère soit vraiment universel.
Une étape importante a donc été de vérifier expérimentalement les prédictions que nous avons pu tirer de cette hypothèse.

C'est ce qui a été abordé au cours du Chapitre~\ref{chap:experiments} de ce manuscrit.
Nous avons mis en place un cadre d'étude expérimental dans lequel la distribution des ressources peut être observée lors d'une transition de croissance.
Tout d'abord, une souche d'\textit{Escherichia coli} a été modifiée en attachant une protéine fluorescente (GFP) à l'une des sous-unités du ribosome.
Cette souche a ensuite été observée en utilisant une technique avancée de microfluidique permettant de réaliser des transitions robustes et contrôlées du milieu de culture.
Enfin, les mesures obtenues ont pu être exploitées \textit{via} le développement d'une technique de reconstruction du signal utilisant le lissage de Kalman.
Plusieurs améliorations du dispositif seront cependant nécessaires pour conclure définitivement sur l'existence d'une production tout-ou-rien des ribosomes après une transition.
Deux scénarios sont cependant envisageables.

Dans le premier cas, le caractère tout-ou-rien ou du moins oscillatoire de la synthèse des ribosomes pourrait être confirmé.
Cela suggèrerait que les mécanismes de régulation de la distribution des ressources maximisent la biomasse produite lors d'une transition, et donc que les microorganismes sont effectivement adaptés à des environnements changeants.
Pour confirmer ces résultats, de nouvelles données devront nécessairement être acquises, notamment en mesurant des transitions similaires dans une variété plus large de conditions environnementales (changement de source de carbone, ajout d'acides aminés dans le milieu pour contourner le métabolisme, utilisation d'une restriction en azote au lieu du carbone, ...).
Le dispositif expérimental pourrait également être modifié pour contrôler un plus grand nombre de paramètres.
Par exemple, en attachant un autre rapporteur fluorescent à une enzyme clé du métabolisme, nous pourrions vérifier si les synthèses de la machinerie d'expression génique et de la machinerie métabolique sont effectivement en anti-phase lors des oscillations.

Dans le second cas, le comportement observé serait différent d'une production tout-ou-rien.
Par construction, cela signifierait qu'une hypothèse du modèle n'est pas respectée.
Par exemple, cela pourrait signifier que l'optimisation de la biomasse produite lors d'une transition n'est pas un objectif pour la cellule.
Une autre possibilité serait que les coûts pour la cellule d'une telle régulation dépassent les éventuels bénéfices qu'elle peut en tirer.
En effet, en ne comparant les schémas de régulation que par leurs bénéfices, nous avons fait l'hypothèse implicite que leurs coûts pour la cellule sont négligeables, sinon au moins comparables.
Il est possible que même si la production de biomasse s'avère être un critère déterminant, les limites physiques imposées par le coût des régulations empêchent la cellule de faire mieux que le comportement observé.
La prise en compte de ces coûts dans le modèle devra donc être explorée avant toute conclusion.

Dans tous les cas, cette exploration des coûts des schémas de régulation pourrait s'avérer cruciale dans l'établissement d'un lien fondamental entre la complexité des schémas de régulation d'une espèce microbienne, et la dynamique de son environnement.
En effet, comme montré dans le Chapitre~\ref{chap:theory}, la cellule ne tire avantage d'un schéma de régulation complexe que lorsque son environnement varie.
On peut élargir ce résultat en supposant que le bénéfice apporté par des régulations toujours plus complexes finit par saturer.
En revanche, le coût de ces régulations devrait continuer à croître, dans la mesure où des régulations plus élaborées impliquent pour la cellule la synthèse de davantage de systèmes de mesure, et donc détournent une part toujours plus grande des ressources cellulaires.
On peut donc conjecturer que ces deux tendances vont finir par se croiser au niveau d'un goulot d'étranglement évolutif, au delà duquel les coûts excèdent les bénéfices.
On peut de plus s'attendre à ce que ce goulot d'étranglement survienne plus ou moins loin en fonction de l'intensité et la fréquence des variations environnementales auxquelles les microorganismes sont soumis.
En d'autres termes, là où les lois de croissance actuelles décrivent uniquement des variations physiologiques avec la qualité du milieu, nous pourrions ajouter une autre dimension à ces lois qui représenterait la variabilité temporelle de la disponibilité des nutriments dans l'environnement naturel.
\end{chapter_summary}
\selectlanguage{english}

\begin{center}
\noindent\rule{4cm}{0.1pt}
\end{center}

\section*{Beginning of Chapter \thechapter}

Physics has yet to be reduced to a formula that will fit on a piece of clothing~\cite{falk_universe_2005}.
This is even more true for Biology, but the extensive work on the mechanisms of evolution are helping to close this gap (see for instance the Price equation, which summarize in a short and elegant formula the mechanisms of evolution and natural selection~\cite{frank_natural_2012}).
But evolution by itself is not sufficient to define life.
Even the least restrictive definition we have -- the one we use to look for life in the universe -- defines a living system as "a self-sustaining chemical system capable of Darwinian evolution"~\cite{deamer_origins_1994,benner_defining_2010}.
Self-sustainment appears to be as fundamental as evolution, and relies on the transformation of matter and energy from the environment into organic matter, in other words, on growth.
But the mathematical formulation of questions regarding the self-sustaining character of living systems has not advanced as much as the mathematical analysis of evolution.

Nevertheless, fundamental growth laws have been established for microorganisms.
These growth laws show that, regardless of the molecular mechanisms underlying growth control, microorganisms tend to follow the same empirical regularities when grown at steady state in different environmental conditions~\cite{molenaar_shifts_2009,scott_emergence_2014,scott_interdependence_2010,scott_bacterial_2011}.
In particular, they adjust their internal molecular composition after a change in the environment in such a way as to maximize their growth rate~(see in particular~\cite{molenaar_shifts_2009,scott_emergence_2014}).
Growth laws are a strong support for the theory of a modular organization of microorganisms~\cite{scott_emergence_2014,hartwell_molecular_1999,arkin_fast_2006,guido_bottom-up_2006}.
They represent a big step forward towards gaining a general understanding of the physiology of microorganisms.
They have been established at steady state, however, a state in which most microorganisms spend very little time~\cite{mcarthur_microbial_2006,menge_nitrogen_2012,
hobbie_microbes_2013,savageau_escherichia_1983,
savageau_demand_1998,blount_unexhausted_2015,vanelsas_survival_2011}.
Why would microorganisms be optimal for a situation they rarely encounter?
Would known growth laws be specific cases of more general laws allowing microorganisms to respond to changes in the environment?

Our aim in this manuscript has been to establish a theoretical and experimental framework extending growth laws to a dynamical context.
We focused on the growth law describing how ribosome abundance adapts to a change in environment so as to maximize steady-state growth in different media.
Applying the same criterion of growth rate maximization, what is the optimal way to allocate resources during a growth transition between two different environments?
In Chapter~\ref{chap:theory}, we used a simple self-replicator model of resource allocation, and showed that the steady state is invariant over a number of regulatory schemes: the growth rate can be maximized by measuring either the environment or the internal state of the cell.
In both cases, the expression of genes encoding the metabolic and gene expression machineries has to be set to a specific value.
By contrast, biomass maximization during a growth transition requires information about the internal state rather than the environment.
Moreover, expression of the metabolic and gene expression machineries is not set to one specific value, but varies with time in an on-off manner.
What is optimal at steady state and during a growth transition is thus not the same, but this does not mean that a single regulatory system cannot meet both demands.

The model developed here is an instance of proof-of-concept model~\cite{servedio_not_2014}.
It does not necessarily aim at quantitatively predicting or controlling the behavior of microorganisms.
By using a simple and abstract representation, it provides a convenient and tractable way of evaluating the implications of dynamical optimality, in particular its divergence from steady-state optimality.
Is a dynamical perspective on growth laws necessary?
Do regulatory mechanisms need to differ in a dynamical context?
Those are the questions that were investigated by means of this model. 
To our surprise, it additionally provided a new way of looking at the regulatory systems controlling ribosomal abundance in many bacteria.

Some readers may be skeptical about the interest of using such a simple model for addressing the above questions.
Given the profusion of knowledge and data available on biological systems, one might be tempted to include in the model every known detail about the molecular implementation of the system of interest.
One of the main drawbacks of such mechanistic models is that we quickly loose track of the big picture and the underlying assumptions.
When including many molecular details, the models also tend to become untractable, prohibiting the use of mathematical tools like optimal control theory that are quickly overwhelmed by the sheer number of variables and parameters in the model.
Besides, simple, abstract models make it easy for the modeler to state the assumptions that are made, and more importantly to identify which ones are logistical, exploratory, or critical (see~\cite{servedio_not_2014}, in particular Box~1).
Logistical assumptions do not affect the conclusions of the model, and are only necessary for tractability.
A good example in our case is the Michaelis-Menten kinetics that were considered in Chapter~\ref{chap:theory}.
Exploratory assumptions, however, might be important to vary and test.
For instance, an exploratory assumption of our model was to consider that the proposed strategies had to be optimal at steady state as well.
This was used as a necessary step to reduce the space of possible solutions, and make the comparison between different strategies time-independent.
Relaxing this assumption would require the exploration of other environmental changes beyond a simple nutrient upshift, but would probably raise interesting considerations~\cite{geisel_constitutive_2011,lopez-maury_tuning_2008,lambert_memory_2014,kussell_phenotypic_2005}.

But the assumptions that are most in need of discussion are the critical assumptions, \textit{i.e.} the ones that are invalidated if the model predictions are not met in nature.
In our case, a critical assumption is that biomass production is the main fitness factor in stable or changing environments.
Throughout the manuscript, we did not really challenge this hypothesis, but only provided arguments about why this assumption is reasonable in some experimental conditions.
The arguments given, however, are mostly based on results obtained with laboratory strains in laboratory conditions.
While it is clear that microorganisms can be cultivated and are even naturally found in conditions where maximizing the growth rate ensures their persistence~\cite{edwards_silico_2001,ibarra_escherichia_2002,lewis_omic_2010,molenaar_shifts_2009}, we have no direct proof that biomass production is the fitness factor that allowed them to naturally persist on evolutionary time-scales.
In addition, one must be careful when explaining the behavior of living systems solely as a consequence of natural selection.
Such systems are embedded in the physical world, and sometimes universal laws just result from the physical limitations that apply to the system (\textit{e.g.}, Monod's law described in Chapter~\ref{chap:introduction}).

The second part of this manuscript was dedicated to setting up an experimental framework for studying resource allocation during growth transitions.
Much emphasis has been given to the experimental issues encountered when studying growth transitions and to come up with possible solutions.
Fluorescent reporter proteins were used to cope with the need for \textit{in-vivo} measurements at high temporal resolution.
Advances in microfluidics were exploited to perform steady-state-to-steady-state transitions following a change in medium, in order to buffer the effect of the pre-culture history~\cite{ng_damage_1962,dufrenne_effect_1997,shaw_effect_1967}.
Finally, the signal of interest was reconstructed through the use of Kalman smoothing, a powerful signal processing algorithm that has, as we showed, many features that make it suitable for the analysis of microscopy time-series data~\cite{kailath_linear_2000,jazwinski_stochastic_2007,kalman_new_1960}.
Overall, the set-up still needs many improvements, though most of them are experimental issues that should be easily fixable.

Chapter~\ref{chap:experiments} must not be taken as a "test of the model" presented in Chapter~\ref{chap:theory}, but rather as a test of its assumptions (see Box~2 in~\cite{servedio_not_2014}).
The model is correct in that bang-bang resource allocation follows mathematically from the assumption that microorganisms maximize their biomass production at steady state and during growth transitions.
The model predictions can be used, however, as a tool to test whether this critical assumption is valid.
In Chapter~\ref{chap:experiments}, we thus apply the experimental set-up to observe the resource allocation profile following a nutrient upshift in \textit{E.~coli}.
While we were not able to arrive at an unambiguous conclusion about the correctness of the model prediction in this manuscript, we can draw up two possible scenarios in the wake of future improvements of the experimental set-up.

In one scenario, the bang-bang or at least oscillatory nature of resource allocation during a growth transition is confirmed.
This would suggest that the regulatory mechanisms of resource allocation do maximize biomass production during growth transitions, hence that microorganisms are adapted for changing environments.
In order to further establish these results, we would need additional data for a broader range of environmental conditions.
The most straightforward extension would be to test other upshift and downshift schemes (different carbon sources, amino-acids supplementation, nitrogen instead of carbon limitation, ...).
The experimental set-up could also be slightly modified to test a broader range of variables.
For instance, the strain could be modified so as to express an additional fluorescent protein that would report the abundance of a key enzyme of metabolism.
Along with the ribosomal reporter, this would allow us to test the prediction that the metabolic and gene expression machineries are expressed in anti-phase.

In the other scenario, we do not observe any trace of a bang-bang resource allocation scheme.
In that case, the crucial questions is: "Which critical assumptions are not valid?"
We cannot discard the possibility that an assumption that was originally identified as exploratory or logistical might in fact be critical for the predictions of the model.
Although unlikely \textit{a priori}, this might be the case for the expressions used for the macroreactions in the model (Eqs~\ref{eq:metaflux}-\ref{eq:machflux}), the definition of the volume as invariably proportional to the total mass of macromolecules (Eq.~\ref{eq:voldef}), or the fact that degradation is assumed negligible with respect to the rates of other reactions in the system.
More likely, the assumptions regarding the cost of the regulatory system could be more critical than initially thought.
By only comparing the regulatory schemes with respect to their benefits, we implicitly made the assumption that their costs are negligible or at least comparable.
The costs of regulatory schemes are difficult to model~\cite{shachrai_cost_2010,dong_gratuitous_1995,
dekel_environmental_2005} since the cellular variables are not equally costly to measure and, as we showed for the ppGpp system, one mechanism may measure several variables at once.
Nevertheless, relaxation of this hypothesis would definitively need to be explored before concluding that microorganisms do not optimize their biomass production during growth transitions.

In any case, exploring the costs of regulatory mechanisms might uncover a fundamental link between the complexity of the regulatory schemes and the dynamics of the environment.
As we illustrated in Chapter~\ref{chap:theory}, complex regulatory schemes only prove beneficial in a dynamical context, away from steady state.
At the opposite extreme, one can expect the benefits of these regulatory schemes to saturate, whereas the cost of complexity would probably not saturate, since acquiring more information is expected to require additional sensing systems, diverting away an increasingly larger part of cellular resources.
Overall, these two tendencies are expected to cross at a point that would represent an evolutionary bottleneck, beyond which the costs exceed the benefits (see~\cite{short_flows_2006} for another example of such a bottleneck).
Interestingly, the position of the bottleneck might be dependent on the environment, resulting in a law that would link the complexity of regulatory systems to the dynamics of the environment.
In other words, while the current growth laws consider how the physiology of the cell varies with the nutrient quality of the medium, we could add another dimension representing the time-varying availability of this nutrient in the natural environment.