\selectlanguage{english}
\begin{abstract}
Growth is one of the most fundamental processes of life, and is highly optimized in microorganisms.
It relies on the transformation of matter and energy from the environment into diverse organic structures.
Decades of studies have shown that despite an extreme microbial diversity, this transformation follows general regularities, forming the so-called microbial growth laws.
But those works focused on steady-state growth, despite this condition being irrelevant to the natural growth condition of microorganisms.

Do these laws still hold when considering a dynamical environment?
What does it mean to be optimal during an environmental transition?

This project aims to lay the foundation of dynamical growth laws, more in adequation with the natural habitats of microorganisms and the selective pressures that shaped their current forms.
We develop an intuitive framework by using a self-replicator model that encapsulates the main features of growth through nutrient assimilation.
We formulate the resource allocation problem that arises as an optimal control problem in which the cell has to allocate its resources between gene expression machinery and metabolism.
By applying the Pontryagin's Maximum Principle, the model predicts the optimal dynamical behavior to be a bang-bang allocation between gene expression machinery and metabolism.
Despite not having been observed yet, we show that such behaviors are conceptually feasible with the known regulations of the enterobacteria \textit{Escherichia coli}, in particular the ppGpp system that regulates ribosomal abundance.

In order to find such patterns in real cells, we make a strain displaying fluorescent ribosomes.
We observe \textit{E. coli} cells in a microfluidic device and monitor their fluorescence during growth transitions.
Using state-of-the-art techniques of parameter estimation, we can reconstruct the state of the cellular resource allocation controler with a high temporal precision.
Even if preliminary, our experimental results seem to confirm the bang-bang singular nature of resource allocation during a growth transition, as predicted by our model.

Our work might play an important role in the formulation of dynamical fundamental growth laws.
It also has application for biotechnological purposes, since microorganisms and in particular \textit{E. coli}, are more and more used in health and energy industries to produce valuable elements.
\end{abstract}

\selectlanguage{french}
\begin{abstract}
La croissance et la reproduction sont des mécanismes fondamentaux du vivant.
Généralement couplés chez les micro-organismes, ces deux processus reposent sur la transformation des ressources de l'environnement (matière et énergie) en nouvelles structures organiques.
Malgré l'extrême diversité des micro-organismes, certaines caractéristiques clés du contrôle de la croissance semblent être largement répandues, au point que des lois de croissance ont pu être érigées.
Ces dernières montrent que des mécanismes encore plus fondamentaux dictent la répartition des ressources dans les différents sous-systèmes de la cellule, généralement dans un souci d'optimisation de son taux de croissance.
Mais ces lois ont été établies uniquement lors d'une croissance à l'état stationnaire, c'est-à-dire lorsque l'environnement et le taux de croissance ne varient pas dans le temps.
Ces conditions, même si elles peuvent être reproduites en laboratoire, n'ont rien à voir avec les conditions de vie dans lesquelles les organismes ont évolué durant des milliards d'années.

Ces lois sont-elles toujours valable en environnement dynamique ?
Comment traduire les concepts d'optimalité dans de tels environnements ?

Le but de ce projet est de poser les bases de nouvelles lois de croissance étendues en environnement dynamique, c'est-à-dire plus proche du milieu réel de vie, et tenant mieux compte des pressions de sélection qui ont donné lieu à ces organismes.
Notre cadre d'étude s'articule autour d'un modèle d'auto-réplicateur qui prend en compte les caractéristiques principales de la croissance par assimilation de nutriments.
De ce cadre émerge un problème d'allocation des ressources entre la machinerie d'expression génique et le métabolisme, que nous formulons comme un problème de contrôle optimal.
Le modèle prédit ainsi via l'application du Principe de Maximisation de Pontryagin que le meilleur comportement possible lors d'une transition est une allocation de type bang-bang, c'est-à-dire lors de laquelle les ressources vont par alternance soit uniquement dans la machinerie, soit uniquement dans le métabolisme.
Malgré l'absence d'observation claire d'une tel phénomène, nous montrons qu'un tel comportement est conceptuellement possible à partir des régulations connues de l'enterobactérie \textit{Escherichia coli}, notamment le système ppGpp qui régule l'abondance des ribosomes.

Afin d'observer de tester ces prédictions, nous avons ingéniéré une souche de \textit{E. coli} arborant des ribosomes fluorescents.
Nous avons observé ces cellules dans un système de microfluidique et enregistré leur fluorescence au cours de transitions de croissance.
En utilisant les dernières méthodes d'identification paramétrique, nous pouvons reconstruire l'état du contrôleur d'allocation des ressources avec une très bonne précision temporelle.
Même s'ils ne sont que préliminaires, ces résultats expérimentaux semblent confirmer la nature bang-bang de l'allocation lors des transitions de croissance chez \textit{E. coli}.

Notre travail peut jouer un rôle clé dans la formalution de nouvelles lois de croissances, étendues en environnement dynamique.
Il a également de nombreuses applications en biotechnologie, dans la mesure où les micro-organismes et plus particulièrement \textit{E. coli} sont très largement utilisés par de nombreuses industries pour produire des molécules valorisables.
\end{abstract}
\selectlanguage{english}
