\selectlanguage{english}
%%% MAX 4000 CHARACTERS %%%%
\begin{abstract}
Growth is the most fundamental property of life.
Growth consists in the transformation of matter and energy from the environment into diverse organic structures.
Interestingly, general growth laws relate the macromolecular composition of the cell to growth rate.
These laws are widespread and conserved in different microbial species, suggesting a fundamental principle of design.
Recent work has shown that these empirical regularities can be derived from coarse-grained models of resource allocation and explained by the principles of natural selection.
However, the vast majority of these studies focus on steady-state growth.
Such conditions are rarely found in natural habitats, where microorganisms are continually challenged by environmental fluctuations.
The aim of this thesis is to extend the theoretical and experimental studies of microbial growth strategies to changing environments.

Using a self-replicator model, we developed a theoretical framework that encapsulates the main features of growth.
We formulate dynamical growth maximization as an optimal control problem that the microbial cell must solve in order to allocate the available resources to the gene expression machinery or to metabolism.
Using Pontryagin's Maximum Principle, we have derived a general solution to the optimization problem and we have compared the optimal strategy with possible implementations of growth control in bacterial cells.
Our results show that simple control strategies that maximize the growth-rate at steady state are suboptimal for transitions from one growth regime to another.
We show that a near-optimal control strategy in dynamical conditions requires information about several, rather than a single, physiological variable.
Interestingly, this strategy has structural analogies with the regulation of ribosomal protein synthesis by the signaling molecule ppGpp in the enterobacterium \textit{Escherichia coli}.
The strategy involves sensing a discrepancy between the concentrations of precursor metabolites and ribosomes, and the control of the rate of ribosome synthesis in a switch-like manner.

Even though this switch-like ribosome synthesis has been suggested by published data, the phenomenon has never been experimentally confirmed.
We therefore measured ribosomal abundance in \textit{Escherichia coli} at the single-cell level during a nutrient upshift.
More precisely, we constructed a strain in which a fluorescent marker has been attached to a ribosomal subunit, thus allowing \textit{in-vivo} monitoring of the abundance of ribosomes.
We monitored this strain in a microfluidics device designed for long-term imaging of individual cells in a continuous culture, and used this experimental setup to simulate a nutrient upshift by changing the input medium.
We developed a Kalman smoothing method for extracting quantitative information about resource allocation to ribosome synthesis from the raw data.
Even though our preliminary results do not allow to reach a final conclusion, they do suggest the presence of oscillatory patterns after an upshift that are reminiscent of the expected behavior.

Our results demonstrate that the capability of regulatory systems to integrate information about several physiological variables is critical for optimizing growth in a changing environment.
The proposed control scheme correctly reproduces the observed growth laws at steady state, but also predicts novel and unexpected behaviors when applied to a dynamical environment.
Our improved understanding of the principles that govern the control of bacterial growth could be used for improving biotechnological processes, in particular those that use microorganisms to produce high valuable-added products for the chemical or biomedical industry.
\end{abstract}

%TC:ignore  

%%% MAX 1000 CHARACTERS %%%
\begin{author-summary}
Bacteria grow by exploiting matter and energy extracted from their environment.
The strategies that enable bacteria to optimize their growth rate have been extensively studied.
However, most of these studies were carried out in a constant environment.
Here, we construct a simple model of microbial growth and use mathematical methods from optimal control theory to determine how the cells should behave when the environment changes abruptly.
We find that microbial cells should adopt an \textit{on-off} strategy:
they should alternatively allocate all of their resources to the production of two categories of cellular components, the elements needed for gene expression and the ones needed for the general metabolism of the cell.
Our preliminary experiments confirm this prediction.
In other words, theory tells us that switch-like mechanisms would optimize the growth of bacterial cells and our experiments validate the theory.
This suggests new ways for optimizing biotechnological processes.
\end{author-summary}

\selectlanguage{french}
%%% MAX 4000 CARACTERES %%%%
\begin{abstract}
Croissance et reproduction sont des mécanismes fondamentaux du vivant.
Chez les micro-organismes, ces processus sont couplés dans la transformation des ressources de l'environnement (matière et énergie) en nouvelles structures organiques.
Étonnamment, malgré l'extrême diversité des micro-organismes, certaines caractéristiques de leur physiologie suivent des lois de croissance universelles.
Cela suggère l'existence de principes fondamentaux, ce qui a été en effet récemment confirmé en montrant que ces lois s'expliquent facilement si l'organisme maximise son taux de croissance dans chaque environnement.
Cependant, ces lois ont seulement été étudiées lors de croissances à l'état stationnaire, c'est-à-dire lorsque l'environnement et donc le taux de croissance sont stables.
Ces conditions, même si elles peuvent être reproduites en laboratoire, n'ont rien à voir avec les conditions de vie dans lesquelles les organismes ont évolué durant des milliards d'années.
Le but de cette thèse est d'étendre dans un contexte d'environnement purement dynamique, l'étude à la fois théorique et expérimentale de ces stratégies de croissance microbienne.

En modélisant la cellule comme un auto-réplicateur, nous cherchons à savoir quelles sont les meilleures stratégies d'allocation des ressources lors de l'adaptation à un nouvel environnement.
Ce problème se formule très bien comme un problème de contrôle optimal : la cellule choisit en temps réel comment allouer ses ressources entre la machinerie d'expression génique et le métabolisme.
Le meilleur comportement possible, comme le révèle l'application du Principe de Maximisation de Pontryagin, est de successivement orienter toutes les ressources dans chacun des deux secteurs, une stratégie communément appelée \textit{bang-bang}.
Mais s'approcher d'un tel contrôle requiert pour la cellule des stratégies de régulation bien plus complexes que celles qui étaient suffisantes pour maximiser le taux de croissance à l'état stationnaire.
De manière intéressante, la régulation de la synthèse des ribosomes par le ppGpp chez la bactérie \textit{Escherichia coli} s'avère présenter la structure adéquate.
Nous montrons en effet qu'elle permet de détecter rapidement toute incompatibilité entre la concentration de précurseurs et celle des ribosomes, et d'ajuster en conséquence la synthèse de ces derniers pour obtenir un comportement proche de l'optimum mathématique prédit.

Même si de vieilles données le suggèrent, un tel comportement \textit{bang-bang} n'a jamais été totalement confirmé expérimentalement pour la synthèse des ribosomes.
Nous mesurons donc l'abondance des ribosomes chez \textit{Escherichia coli} au niveau d'une cellule unique lors d'un changement brutal de milieu de culture.
En particulier, nous créons une souche de \textit{E.coli} sur laquelle un rapporteur fluorescent est attaché à l'une des sous-unités ribosomales, permettant ainsi leur quantification \textit{in vivo}.
Un appareillage micro-fluidique nous permet ensuite de contrôler en temps réel le milieu de croissance tout en observant individuellement chaque cellule fluorescente.
Nous développons une méthode basée sur le lissage de Kalman qui est capable de reconstruire la façon dont les ressources sont aiguillées vers la synthèse des ribosomes.
Même si ces résultats sont préliminaires, ils suggèrent que la concentration des ribosomes oscille après le changement d'environnement, ce qui rappelle une stratégie de type \textit{bang-bang}.

Nos résultats montrent que la capacité des systèmes de régulations à intégrer l'état de différentes variables physiologiques est crucial dans l'optimisation de la croissance en environnement variable.
Au final, nous démontrons que les principes utilisés à l'état stationnaire peuvent, lorsqu'ils sont appliqués en dynamique, générer des comportements inattendus et expliquer plus en détails les stratégies de régulations employées par les micro-organismes.
\end{abstract}

\begin{resume-vulgaire}
Bactéries et autres micro-organismes se multiplient en utilisant la matière et l'énergie présentent dans leur environnement.
Les stratégies qui leur permettent d'optimiser ce processus ont été longuement étudiées, mais uniquement dans des environnements stables.
Ici, nous construisons un modèle simple de la croissance microbienne et utilisons la théorie du contrôle optimal pour déterminer la façon dont les cellules devrait agir lorsque leur environnement est soudainement modifié.
Nous mettons en évidence que la meilleure stratégie consiste à aiguiller toutes les ressources disponibles vers la production d'un seul composant de manière alternative, une prédiction que nous tentons également de confirmer de manière expérimentale.
En d'autres termes, dans un contexte dynamique, mieux vaut agir tel un interrupteur à bascule plutôt qu'un variateur de lumière.
Ces résultats pourraient suggérer de nouvelles façons d'optimiser les processus biotechnologiques de production de molécules d'intérêt.
\end{resume-vulgaire}

\selectlanguage{english}
%TC:endignore  