\cleardoublepage
\selectlanguage{french}
\begin{acknowledgements}

Il est probable que cette page, même si elle s'adresse au public restreint des francophones, sera la plus consultée de cette thèse.
Cela fait effectivement longtemps que le manuscrit de thèse a perdu le monopole de la communication scientifique.
On exige cependant qu'il soit traité avec le même respect qu'il y a un siècle, en faisant comme s'il n'avait rien perdu de sa gloire passée, non sans une certaine forme d'hypocrisie administrative.
Ce paradoxe rend généralement cet exercice difficile, mais par sa longueur, le déroulement du doctorat lui même l'est bien davantage.
À ce jour, j'ai par exemple investi plus de 15\% de mon existence dans ce projet de recherche.
C'est long, surtout pour quelqu'un comme moi qui aime toucher à tout.
Inévitablement, de nombreuses rencontres qui ont grandement influencé ce travail se sont succédées, et c'est le rôle de cette page que de leur manifester ma gratitude.

Tout d'abord, je remercie évidemment Marion Joulié, qui même si son engagement date de bien avant ce doctorat, a su rester jusqu'au bout et restera encore je l'espère pour les nombreuses années à venir.

Je remercie particulièrement Valentin Zulkower et Edith Grac qui étaient présents pendant la meilleure période de ce doctorat.
Sans nos nombreuses discussions, ce travail n'aurait probablement pas été le même.
Viennent ensuite tous les jeunes membres des équipes Ibis et BIOP, notamment les passages remarqués de Çiğdem Ak, Cindy Barillot, Diana Stefan, Bernard Chelli, Ludowic Lancelot, Julien Sauvage, et tous les autres dont les noms ne me viennent pas au moment d'écrire ces lignes.
Je n'oublie pas de remercier les (nombreux) jeunes chercheurs de l'équipe Mistis qui ont su m'accueillir lorsque la plupart des gens sus-cités se sont progressivement évaporés.

D'un point de vue plus scientifique, je remercie évidemment mes directeurs Hans Geiselmann et Hidde de Jong, mais aussi Corinne Pinel, Irina Mihalcescu, Delphine Ropers, Eugenio Cinquemani, Jean-Luc Gouzé, et Francis Mairet, qui ont apporté des contributions plus que concrètes dans les étapes clés de ce travail.
Je remercie pour les mêmes raisons tous les membres des équipes Ibis et BIOP avec qui j'ai pu échanger lors de nos nombreuses réunions hebdomadaires.

Enfin, je tenais également à remercier les communautés de Bioinfo-fr.net et JeBiF -- RSG France, en particulier jnsll pour ses corrections des parties françaises de ce manuscrit, mais aussi Chopopope, Kumquatum, Norore, Hautbit, m4rsu et Will qui ont su virtuellement me supporter dans les meilleures et les pires périodes de ce doctorat.
\end{acknowledgements}
\selectlanguage{english}